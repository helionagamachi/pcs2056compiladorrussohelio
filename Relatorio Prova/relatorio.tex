\documentclass[12pt,a4paper,titlepage]{article}
\usepackage[latin1]{inputenc}
\usepackage[brazil]{babel}

\usepackage[T1]{fontenc} % permite hifeniza��o em palavras acentuadas
\usepackage[scaled]{helvet} % usa uma fonte sem serifa no tamanho adequado
\renewcommand*\familydefault{\sfdefault} % usa a fonte n�o serifada

\usepackage{ae}
\usepackage{graphicx} % para poder inserir imagens PNG, JPG e GIG
\usepackage{verbatim}
\usepackage{rotating}
\usepackage{fullpage}
\usepackage{indentfirst}
\usepackage{moreverb}
\newcommand{\LinhaHorizontal}{\rule{\linewidth}{0.5mm}}
\usepackage[show]{ed} % para adicionar notas de edi��o - retirar da vers�o final (usar o comando \ednote{})
\def\verbatimtabsize{2\relax}
\usepackage{color}
	\definecolor{black}{RGB}{0, 0, 0}
\usepackage[breaklinks]{hyperref}
\hypersetup{
    pdfnewwindow=true,			% links in new window
    colorlinks=true,			% false: boxed links; true: colored links
    linkcolor=black,	% color of internal links
    citecolor=black,			% color of links to bibliography
    filecolor=black,			% color of file links
    urlcolor=black				% color of external links
}


\usepackage{verbatim}
% \def\verbatimtabsize{2\relax}

\usepackage{multirow}

%C�DIGOS FONTE
\usepackage{listings} % para importa��o de c�digos fonte
\renewcommand{\lstlistingname}{C�digo} % defini��o visual dos c�digos fonte
\lstset{
	extendedchars=true,
	inputencoding=latin1,
	belowcaptionskip=5pt,
	numbers=left,
	stepnumber=1,
	firstnumber=1,
	numberstyle=\tiny,
	breaklines=true,
	frame=tb,
	basicstyle=\ttfamily\small,
	showstringspaces=false,
	mathescape,
	tabsize=3,
	% prebreak = \raisebox{0ex}[0ex][0ex]{\ensuremath{\hookleftarrow}},
	keywordstyle=\ttfamily\color[rgb]{0,0,1},
	commentstyle=\ttfamily\color[rgb]{0.133,0.545,0.133},
	stringstyle=\ttfamily\color[rgb]{0.627,0.126,0.941}
}

\newcommand{\inputpng}[4]{%
\begin{figure}[!htb]
  \centering
  \includegraphics[width=#4\textwidth]{#1.png}
  \caption{\it #3.}
  \label{fig:#2}
\end{figure}
}
\begin{document}
	% T�tulo
	\begin{titlepage}
\begin{center}

% Cabe�alho
\href{http://www.poli.usp.br/}{\includegraphics[width = 5.75cm]{figuras/Header}}
\vfill\

% T�tulo
\LinhaHorizontal\
\\[0.8cm]
\begin{Huge}\bfseries{Segunda Prova}\end{Huge}\\[.2cm]
\begin{Large}PCS-2056 : Linguagens e Compiladores\end{Large}
\\[0.4cm]
\LinhaHorizontal\
\vfill\

% Autores e Professor
\begin{minipage}{.4\textwidth}
	\begin{flushleft}
		\emph{Autores:}\\
		\begin{large}Eduardo Russo\\Helio Kazuo Nagamachi\end{large}
	\end{flushleft}
\end{minipage}
\begin{minipage}{.4\textwidth}
	\begin{flushright}
		\emph{Professor:}\\
		\begin{large}Ricardo Rocha\end{large}
	\end{flushright}
\end{minipage}
\\[1cm]

% Data
\begin{large}\today\end{large}

\end{center}
\end{titlepage}
	% Sum�rio
	\tableofcontents
	\newpage
	
	%
	% Cap�tulos
	%

	
	\section{Objetivo}
		Este documento tem como objetivo descrever a constru��o do compilador de Lua -- uma linguagem de programa��o de extens�o -- para um ambiente de execu��o escrito em Java para a segunda prova da disciplina de \textbf{Linguagens e Compiladores}.
	\section{Gram�tica}
		A linguagem Lua apresentada para a prova -- c�digo \ref{code:codigos_Gramatica} -- foi reduzida de forma a necessitar de apenas tr�s Aut�matos de Pilha Estruturado (APEs).
\lstinputlisting[tabsize = 4, caption={Gram�tica.}, label = {code:codigos_Gramatica}]{codigos/Gramatica}

Para a redu��o, foi utilizado um aplicativo desenvolvido em Python para facilitar o trabalho. Esta redu��o � apresentada no c�digo \ref{code:codigos_Gramatica-reduzida}
\lstinputlisting[tabsize = 4, caption={Gram�tica reduzida.}, label = {code:codigos_Gramatica-reduzida}]{codigos/Gramatica-reduzida}
	
	\section{An�lise l�xica}
		Para a an�lise l�xica, foram identificadas as palavras reservadas na linguagem apresentadas a seguir:

\begin{table}[htb!]
	\label{table:label}
	\centering
	\begin{tabular}{p{1.1cm} p{1.1cm} p{1.1cm} p{1.1cm} p{1.1cm} p{1.1cm} p{1.1cm} p{1.1cm} p{1.1cm} p{1.1cm} }
		\#		& - 	& + 		& * 	& / 	& $\wedge$		& \% 	& . 	& .. 	& ... \\
		< 		& <= 	& >			& >= 	& == 	& $\sim$= 		& and 		& or 	&  \& 	& ; \\
		: 		& [ 	& ] 		& = 	& \{ 	& \} 					& ( 		& ) 	& end 	& nil \\
		false 	& true 	& return	& not 	& break & local 				& function	& for 	& in 	& do \\
		if 		& then 	& elseif 	& else	& while & repeat				& until 	& 		&		&
	\end{tabular}
\end{table}

Considerando essas palavras reservadas, mas as poss�veis sequ�ncias de palavras, foram criados aut�matos para reconhecer e retornar os \emph{tokens}.

A constru��o do l�xico utilizou os c�digos apresentados na figura \ref{fig:figuras_lexico}.

\inputpng{figuras/lexico}{figuras_lexico}{C�digos utilizados para construir o analisador l�xico}{1.0}

A classe principal do analisador l�xico, que chama as fun��es dos outros c�digos encontra-se no c�digo \ref{code:anallex}.

\lstinputlisting[language=Java, tabsize = 4, caption={Classe ``Analyzer.java'' que controla o processo de an�lise l�xica.}, label = {code:anallex}]{codigos/Analyzerlex.java}
	\section{Analise sint�tica}
		A an�lise sint�tica se utiliza de um Aut�mato de Pilha Estruturado (APE) que � descrito por \cite{JJ}, estrutura que utiliza uma pilha e alguns aut�matos finitos para realizar a tarefa de an�lise sint�tica.

\subsection{Estrutura}
O analisador sint�tico possu�, basicamente, um APE e acesso a c�digos que permitem o \emph{parse} de arquivos de configura��o dos aut�matos finitos.

% O APE, tem sua estrutura definida por :

\subsubsection{Aut�mato Finito}
O aut�mato finito utilizado para a an�lise sint�tica possu� um vetor de booleanos que denota os seus estados. Se uma posi��o do vetor possuir o booleano ``verdadeiro'', quer dizer que aquele estado � final. H� tamb�m um vetor com as transi��es do aut�mato.

\subsubsection{Transi��o}
A transi��o pode ser de dois tipos: transi��o normal ou a chamada de outro aut�mato.

Se a transi��o for normal, ela possu� o pr�ximo estado e o token que a ativa.

Se a transi��o for uma chamada, ela cont�m o n�mero do aut�mato a ser chamado e o estado de retorno, que � um estado no aut�mato atual que ser� o estado corrente ap�s a execu��o do segundo aut�mato.

\subsubsection{Pilha}
A pilha � a estrutura utilizada pelo APE para armazenar a informa��o de aut�mato e estado quando a transi��o de um aut�mato a ser executada � a chamada de outro aut�mato.

Quando esse tipo de transi��o ocorre, o aut�mato corrente passa a ser o aut�mato denotado pela transi��o e a pilha guarda qual o aut�mato que realizou a chamada e qual o estado que esse deve voltar ap�s a execu��o do segundo aut�mato.

\subsection{Parse}
A gram�tica em nota��o de Wirth foi reduzida de forma a ficarmos com 3 aut�matos: um de programa, um de bloco de c�digo e o terceiro de express�es.

A gram�tica foi transformada em aut�matos finitos determin�sticos m�nimos com o uso da ferramenta criada por Hugo Bara�na e Fabio Yamate \cite{RadiantFire}, um meta compilador de defini��es de gram�tica em nota��o Wirth.

Um exemplo da sa�da para um dos aut�matos encontra-se na se��o de anexo. As transi��es do aut�mato de bloco est�o nos anexos -- c�digo \ref{automato:bloco_codigo}. A primeira linha do arquivo � sempre igual, indicando que o estado inicial do aut�mato � o estado zero, a segunda linha mostra quais s�o os estados finais e as outras linhas indicam as transi��es. Se o elemento da transi��o representar um token � uma transi��o normal, caso seja o nome de um aut�mato, indica a chamada para o outro aut�mato.

Conforme o arquivo de configura��o � lido, descobre-se ao final a quantidade de estados do aut�mato.

\subsection{An�lise Sint�tica}

A an�lise sint�tica se d� com a requisi��o dos tokens um a um do sint�tico para o l�xico.

O analisador gerencia o APE e, na eventualidade de um erro, o processo para e uma mensagem de erro � apresentada. Caso o �tomo que marca o fim do texto fonte seja recebido, o analisador deve verificar se n�o h� dados na pilha de aut�matos e se o aut�mato atual � o primeiro aut�mato -- no caso o aut�mato referente a programa -- e o estado corrente desse aut�mato � final. Caso essas condi��es sejam atendidas, o programa � v�lido sob o ponto de vista sint�tico.

 
	\section{Defini��o do ambiente de execu��o}
		O ambiente de execu��o � composto por uma cole��o de bibliotecas escritas em baixo n�vel, assembly da MVN para prover algumas funcionalidades b�sicas. S�o elas:

\begin{itemize}
	\item fun��o l�gica OU
	\item fun��o l�gica E
	\item fun��o comparativa >
	\item fun��o comparativa <
	\item fun��o comparativa >=
	\item fun��o comparativa <=
	\item fun��o comparativa ==
	\item fun��o comparativa !=
	\item fun��o l�gica !
	\item fun��o print de strings
\end{itemize}

A ideia � manter esses arquivos com essas fun��es em arquivos separados, pelo uso da biblioteca do montador, ligador e relocador, um �nico arquivo do programa principal pode ser gerado.

\subsection{Registros de Ativa��o}
A implementa��o da MVN n�o tem uma pilha em seu ambiente de execu��o, isso implica na impossibilidade de realizar chamadas de fun��o recursivamente. Um modo de fazer isso � utilizando-se de registros de ativa��o para contornar o problema.

	\section{Estrutura de dados e algoritmos do compilador}
		\input{60-estrutura_de_dados}
	\section{Sem�ntica din�mica}
		Na an�lise sem�ntica � realizada a gera��o de c�digo e outras verifica��es que n�o podem ser feitas apenas com a defini��o da gram�tica

\subsection{Introdu��o das a��es sem�nticas}
As a��es sem�nticas foram introduzidas na gram�tica, em sua nota��o de Wirth. Para podermos continuar a utilizar a ferramenta 

\subsection{Estruturas}
Para algumas a��es sem�nticas foram criadas estruturas de dados espec�ficas para auxiliar na sua fun��o.

\subsection{A��es}
Uma a��o sem�ntica � executada quando o ocorre a transi��o a que ela est� associada. Por necessidade, cada aut�mato, pode ter uma a��o final que � executada apenas quando o aut�mato ir� retornar para outro aut�mato.

\subsubsection{push\_term}
A��o Sem�ntica da m�quina de express�o que empilha o operando na pilha de operandos.

\subsubsection{push\_op}
A��o sem�ntica da m�quina de express�o que empilha o operador na pilha de operadores, faz verifica��o com o topo atual da pilha para que as opera��es aritm�ticas possam ser realizadas de acordo com sua prioridade.

Seu funcionamento � enumerado a seguir:

\begin{enumerate}
	\item Verifica o topo da pilha de operadores.
	\item Se o operador que est� no topo da pilha � mais priorit�rio do que o novo operador, desempilha dois operandos e o operador, gera o c�digo da opera��o e gera uma \emph{label} tempor�ria que ir� receber o resultado da opera��o realizada. Essa \emph{label} � empilhada na pilha de operandos e volta para o passo inicial.
	\item Se o topo est� vazio, ou o operador que est� no topo tem prioridade menor ou igual que o novo operador, simplesmente empilha o novo operador.
\end{enumerate}

\subsubsection{Express�o - a��o final}
Quando a m�quina de express�o deve retornar de sua chamada, deve, tamb�m, gerar o c�digo correspondente � express�o presente na pilha. A gera��o de c�digo � bastante simples, uma vez que a complexidade de lidar com a preced�ncia de opera��es j� foi resolvida. Essa a��o p�ra no momento em que um token de abertura de par�nteses aparece na pilha de operandos, indicando que a a��o deve simplesmente guardar o resultado em uma vari�vel tempor�ria para outro processamento de express�o. Essa vari�vel tempor�ria � empilhada na pilha de operandos e, se n�o houver mais dados para serem processados, o resultado � armazenado em uma nova vari�vel e tamb�m estar� � disposi��o para outros comandos no acumulador.

\subsubsection{nova\_var}
A��o sem�ntica que associa uma nova \emph{label} � variavel criada.

\subsubsection{atribuicao}
Atribui um valor a vari�vel que est� � esquerda no comando, atualmente apenas inteiros, booleanos e caracters podem ser atribu�dos.

\subsubsection{fecha\_escopo}
Emite uma \emph{label} para o final do escopo corrente, no caso de \emph{while} e \emph{else}, eles podem emitir uma instru��o que leva at� essa \emph{label} antes mesmo dela ser sido declarada, por isso, deve verificar se h� alguma desse tipo presente em uma pequena pilha para essa finalidade.

\subsubsection{novo\_escopo}
Emite uma \emph{label} correspondente ao novo escopo. Pode ser utilizada pela instru��o \emph{while}, que, ao chegar ao seu final, faz um desvio incondicional para essa \emph{label}.
	\section{Integra��o das rotinas sem�nticas}
	 	A implementa��o do compilador permite ao usu�rio as funcionalidades de mostrar na tela um inteiro de que esteja contido em uma vari�vel, express�es aritm�ticas, considerando prioridade e express�es booleanas, sem considera��o a respeito das express�es boolenas.

Infelizmente as bibliotecas do montador, relocador e lingador apresentam algum erro em sua implementa��o e acabam inserindo no c�digo bin�rio final no meio do arquivo a linha \textbf{0000 0002}, um desvio incondicional para o endere�o 2 de mem�ria, arruinando os resultados.

	
	% Refer�ncias
	% \nocite{*} % mostrar todos, n�o s� os citados
	% \bibliographystyle{apalike} % estilo da ABNT
	% \bibliography{bibliografia} % arquivo bibtex
	% \addcontentsline{toc}{section}{Refer�ncias}
\end{document}