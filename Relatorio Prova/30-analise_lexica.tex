Para a an�lise l�xica, foram identificadas as palavras reservadas na linguagem apresentadas a seguir:

\begin{table}[htb!]
	\label{table:label}
	\centering
	\begin{tabular}{p{2cm} p{2cm} p{2cm} p{2cm} p{2cm} p{2cm}}
		\#			& - 	& + 		& * 	& / 	& $\wedge$	\\
					&		&			&		&		&			\\
		< 			& <= 	& >			& >= 	& == 	& $\sim$=	\\
					&		&			&		&		&			\\
		: 			& [ 	& ] 		& = 	& \{ 	& \}		\\
					&		&			&		&		&			\\
		false 		& true 	& return	& not 	& break & local		\\
					&		&			&		&		&			\\
		if 			& then 	& elseif 	& else	& while & repeat	\\
					&		&			&		&		&			\\
		\% 			& . 	& .. 		& ... 	& and 	& or		\\
					&		&			&		&		&			\\
		\& 			& ; 	& ( 		& ) 	& end 	& nil 		\\
					&		&			&		&		&			\\
		function	& for 	& in		& do 	&until				\\
	\end{tabular}
\end{table}

Considerando essas palavras reservadas, mas as poss�veis sequ�ncias de palavras, foram criados aut�matos para reconhecer e retornar os \emph{tokens}. Seus tipos s�o apresentados no c�digo \ref{code:codigos_TokenType}.

\lstinputlisting[language=Java, tabsize = 4, caption={\emph{Tipos de tokens definidos: palavra reservada, identificador, n�mero, final de arquivo e cadeia de palavras.}}, label = {code:codigos_TokenType}]{codigos/TokenType.java}


A constru��o do l�xico utilizou os c�digos apresentados na figura \ref{fig:figuras_lexico} e sua classe principal -- que chama as fun��es dos outros c�digos -- encontra-se no c�digo \ref{code:anallex}.

\inputpng{figuras/lexico}{figuras_lexico}{C�digos utilizados para construir o analisador l�xico}{1.0}

\lstinputlisting[language=Java, tabsize = 4, caption={\emph{Classe ``Analyzer.java'' do pacote ``lex'', controla o processo de an�lise l�xica.}}, label = {code:anallex}]{codigos/Analyzerlex.java}