Como visto anteriormente, a gram�tica em nota��o de Wirth foi reduzida de forma a ficarmos com 3 automatos, um de ``comandos'', um de ``fun��es'' e o terceiro de ``express�es''.

Esta gram�tica foi transformada em aut�matos finitos determin�sticos m�nimos com o uso da ferramenta desenvolvida por Fabio Sendoda Yamate e Hugo Bara�na\footnote{Encontra-se em http://radiant-fire-72.heroku.com/} que aplica o algoritmo de convers�o e gera os APEs automaticamente.

A tabela de transi��o dos aut�matos encontra-se a seguir\footnote{Os APEs n�o s�o apresentados pois ficaram muito confusos de visualizar.}:

\lstinputlisting[caption={Transi��es do Wirth de ``comandos''.}, label = {code:comando}] {codigos/comando}

\lstinputlisting[caption={Transi��es do Wirth de ``fun��es''.}, label = {code:funcao}] {codigos/funcao}

\lstinputlisting[caption={Transi��es do Wirth de ``express�es''.}, label = {code:expressao}] {codigos/exp}

Para a contru��o do analisador sint�tico, foram usados os c�digos apresentados na figura \ref{fig:figuras_sintatico}.

\inputpng{figuras/sintatico}{figuras_sintatico}{C�digos utilizados para construir o analisador sint�tico}{1.0}

A classe principal do analisador sint�tico, que chama as fun��es dos outros c�digos encontra-se no c�digo \ref{code:analsint}.

\lstinputlisting[language=Java, tabsize = 4, caption={Classe ``Analyzer.java'' que controla o processo de an�lise sint�tica.}, label = {code:analsint}]{codigos/Analyzersint.java}
