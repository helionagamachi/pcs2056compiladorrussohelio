Para a an�lise l�xica, foram identificadas as palavras reservadas na linguagem apresentadas a seguir:

\begin{table}[htb!]
	\label{table:label}
	\centering
	\begin{tabular}{p{1.1cm} p{1.1cm} p{1.1cm} p{1.1cm} p{1.1cm} p{1.1cm} p{1.1cm} p{1.1cm} p{1.1cm} p{1.1cm} }
		\#		& - 	& + 		& * 	& / 	& $\wedge$		& \% 	& . 	& .. 	& ... \\
		< 		& <= 	& >			& >= 	& == 	& $\sim$= 		& and 		& or 	&  \& 	& ; \\
		: 		& [ 	& ] 		& = 	& \{ 	& \} 					& ( 		& ) 	& end 	& nil \\
		false 	& true 	& return	& not 	& break & local 				& function	& for 	& in 	& do \\
		if 		& then 	& elseif 	& else	& while & repeat				& until 	& 		&		&
	\end{tabular}
\end{table}

Considerando essas palavras reservadas, mas as poss�veis sequ�ncias de palavras, foram criados aut�matos para reconhecer e retornar os \emph{tokens}.

A constru��o do l�xico utilizou os c�digos apresentados na figura \ref{fig:figuras_lexico}.

\inputpng{figuras/lexico}{figuras_lexico}{C�digos utilizados para construir o analisador l�xico}{1.0}

A classe principal do analisador l�xico, que chama as fun��es dos outros c�digos encontra-se no c�digo \ref{code:anallex}.

\lstinputlisting[language=Java, tabsize = 4, caption={Classe ``Analyzer.java'' que controla o processo de an�lise l�xica.}, label = {code:anallex}]{codigos/Analyzerlex.java}