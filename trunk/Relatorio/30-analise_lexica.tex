A an�lise l�xica � a primeira verifica��o feita sobre os programas escritos, basicamente separa o texto fonte em �tomos que podem ser utilizadaos nas etapas posteriores da compila��o. Na an�lise l�xica todo o conte�do irrelevante para a compila��o � descartado, como por exemplo: Espa�os em branco, marcadores de nova linha, e coment�rios.
\subsection{Tokens}
A an�lise l�xica do compilador se d� basicamente pelo reconhecimento dos Tokens definidos para a linguagem, s�o eles:
\begin{itemize}
	\item Identificador
	\item N�mero
	\item Palavra Reservada
	\item Caracter
	\item Coment�rio
	\item String
\end{itemize}

Para o reconhecimento dos Tokens(�tomos), lan�ou-se m�o do uso de Aut�matos finitos, que aceitam como entrada caracteres, e estando em um estado final, podem emitir um Token.

O token foi modelado com 3 campos: 

\begin{itemize}
	\item Valor
	\item Tipo
	\item Linha
\end{itemize}

O valor do Token depende diretamente do tipo de token, por exemplo, se o token for do tipo palavra reservada, seu valor � o n�mero correspondente � palavra reservada representada, se for um n�mero o valor assume a parte inteira do n�mero.
A linha foi definida como um inteiro, indica a linha em que o Token aparece no texto fonte para fins informativos, para aux�lio na depura��o pelo usu�rio.
