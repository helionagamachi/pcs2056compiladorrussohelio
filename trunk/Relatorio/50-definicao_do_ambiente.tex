O ambiente de execu��o � composto por uma cole��o de bibliotecas escritas em baixo n�vel, assembly da mvn para prover algumas funcionalidades b�sicas. S�o elas:
\begin{itemize}
	\item fun��o l�gica OU
	\item fun��o l�gica E
	\item fun��o comparativa >
	\item fun��o comparativa <
	\item fun��o comparativa >=
	\item fun��o comparativa <=
	\item fun��o comparativa ==
	\item fun��o comparativa !=
	\item fun��o l�gica !
	\item fun��o print de strings
\end{itemize}

A ideia � manter esses arquivos com essas fun��es em arquivos separados, pelo uso da biblioteca do montador , ligador e relocador, um �nico arquivo do programa principal pode ser gerado.

\subsection{Registros de Ativa��o}
A implementa��o da MVN n�o tem uma pilha em seu ambiente de execu��o, isso implica na impossibilidade de realizar chamadas de fun��o recursivamente, um modo de fazer isso � utilizando-se de registros de ativa��o para contornar o problema.
