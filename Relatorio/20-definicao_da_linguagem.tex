A defini��o da linguagem em nota��o de WIRTH apresenta-se no c�digo \ref{code:grammar} a seguir:
\lstinputlisting[caption={Gram�tica definida.}, label = {code:grammar}] {codigos/grammar} 

Na defini��o da gram�tica n�o h� detalhamento de como s�o os identificadores, n�meros e caracteres. Eles s�o reconhecidos pelo analisador l�xico e j� s�o recebidos pelo analisador sint�tico como tokens -- ou �tomos -- prontos e apropriados, por isso, na gram�tica aparecem como elementos finais.

Uma breve descri��o de como cada um dos elementos citados acima, s�o esperados � mostrado a seguir:

\begin{itemize}
	\item \textbf{Identificador:} pelo menos uma letra, seguida de letras e n�meros. N�o pode ser uma palavra reservada.
	\item \textbf{N�mero:} n�meros inteiros sem sinal, e de ponto flutuante, no entanto como a MVN s� trabalha com inteiros, somente a parte inteira � considerada.
	\item \textbf{Caractere:} uma letra entre aspas simples.
	\item \textbf{String:} Come�a com aspas duplas, uma sequ�ncia qualquer de caracteres e termina com aspas duplas.
\end{itemize}