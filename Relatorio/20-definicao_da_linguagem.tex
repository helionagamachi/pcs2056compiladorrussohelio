A defini��o da linguagem em nota��o de WIRTH segue:
\lstinputlisting[language=none, caption={Gram�tica}, label = {code:grammar}] {codigos/grammar} 

Na defini��o da gram�tica n�o h� detalhamento de como s�o os identificadores , n�meros e caracteres, eles s�o reconhecidos pelo analisador l�xico e j� s�o recebidos pelo analisador sint�tico como tokens prontos e apropriados, por isso na gram�tica aparecem como elementos finais.

Uma breve descri��o de como cada um dos elementos citados acima, s�o esperados
\begin{itemize}
	\item Identificador: Pelo menos uma letra, seguida de letras e n�meros, n�o pode ser uma palavra reservada.
	\item N�mero: N�meros inteiros sem sinal, e de ponto flutuante, no entanto como a mvn s� trabalha com inteiros, somente a parte inteira � considerada
	\item Caractere: uma letra entre aspas simples
	\item String: Come�a com aspas duplas, uma sequencia qualquer de caracteres e termina com aspas duplas.
\end{itemize}