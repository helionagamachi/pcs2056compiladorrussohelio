\documentclass[%ruledheader, % cabe�alho com nome da se��o e linha
               noindentfirst, % n�o identar primeiro par. da se��o
               normaltoc, % tira o p. no in�cio das p�ginas do sum�rio
               normalfigtabnum, % numera��o de figuras dependente do cap�tulo
               %sumarioincompleto,
               %naousarHyperref, % n�o usa o HyperRef
               %naousarRefABNT, % n�o carrega o pacore de refer�ncias da ABNT
               % twoside,
               %times, % usa fontes tipo windows (times new roman, arial, \ldots)
               anapnormal, % palavra anexo n�o aparece em mai�sculo
               % dvipdfm, %essa merda n�o deixava PNG funcionar!
               anapcustomindent % n�o indenta o restante do t�tulo do anexo
							]{abnt-meu}
\usepackage[latin1]{inputenc}
\usepackage{helvet} \renewcommand{\familydefault}{\sfdefault}
\usepackage{ae}
\usepackage{poli} % personaliza��o da ABNT para a Poli
\usepackage{graphicx} % para poder inserir imagens PNG, JPG e GIG
\usepackage{verbatim}
\usepackage{moreverb}
\usepackage[show]{ed} % para adicionar notas de edi��o - retirar da vers�o final (usar o comando \ednote{})
\def\verbatimtabsize{2\relax}

\usepackage{color}
\usepackage{listings} % para c�digos fonte
\renewcommand{\lstlistingname}{C�digo}
\lstset{
	inputencoding=latin1,
	commentstyle=\it,
	stringstyle=\bf,
	belowcaptionskip=5pt,
	numbers=left,
	stepnumber=1,
	firstnumber=1,
	numberstyle=\tiny,
	extendedchars=true,
	breaklines=true,
	frame=tb,
	basicstyle=\footnotesize,
	stringstyle=\ttfamily,
	showstringspaces=false,
	mathescape,
	tabsize=1
}
\begin{document}
\titulo{PCS2046 - LINGUAGENS E COMPILADORES \\ Relat�rio XXX}
\autoresPoli{EDUARDO RUSSO}{H�LIO KAZUO NAGAMACHI}{\ }{\ }{\ }                                                                                                                                 
\local{Professor: Ricardo Rocha} \data{outubro/2010}
\capa
\setcounter{page}{1}

%
% Cap�tulos
%
\tableofcontents
\chapter{Introdu��o}
	Compiladores s�o programas capazes de traduzir o texto fonte de uma linguagem espec�fica para o texto objeto de uma outra linguagem espec�fica, normalmente a segunda linguagem � o formato entendido por um processador para executar programas.

Nas aulas foram vistos m�todos e t�cnicas para analisar sobre o ponto de vista sint�tico o texto fonte de um programa, e t�cnicas para a gera��o de c�digo execut�vel.

O objetivo deste projeto � exercitar os conceitos vistos em aula para modelar e construir um compilador que utilize uma linguagem definida pelos alunos e gere c�digo objeto para a MVN.

\chapter{Defini��o da linguagem de alto n�vel}
	\lstinputlisting[tabsize = 4, caption={Gram�tica.}, label = {code:codigos_Gramatica}]{codigos/Gramatica}

\lstinputlisting[tabsize = 4, caption={Gram�tica reduzida.}, label = {code:codigos_Gramatica-reduida}]{codigos/Gramatica-reduzida}
	
\chapter{Analise l�xica}
	Para a an�lise l�xica, foram identificadas as palavras reservadas na linguagem apresentadas a seguir:

\begin{table}[htb!]
	\label{table:label}
	\centering
	\begin{tabular}{p{2cm} p{2cm} p{2cm} p{2cm} p{2cm} p{2cm}}
		\#			& - 	& + 		& * 	& / 	& $\wedge$	\\
					&		&			&		&		&			\\
		< 			& <= 	& >			& >= 	& == 	& $\sim$=	\\
					&		&			&		&		&			\\
		: 			& [ 	& ] 		& = 	& \{ 	& \}		\\
					&		&			&		&		&			\\
		false 		& true 	& return	& not 	& break & local		\\
					&		&			&		&		&			\\
		if 			& then 	& elseif 	& else	& while & repeat	\\
					&		&			&		&		&			\\
		\% 			& . 	& .. 		& ... 	& and 	& or		\\
					&		&			&		&		&			\\
		\& 			& ; 	& ( 		& ) 	& end 	& nil 		\\
					&		&			&		&		&			\\
		function	& for 	& in		& do 	&until				\\
	\end{tabular}
\end{table}

Considerando essas palavras reservadas, mas as poss�veis sequ�ncias de palavras, foram criados aut�matos para reconhecer e retornar os \emph{tokens}. Seus tipos s�o apresentados no c�digo \ref{code:codigos_TokenType}.

\lstinputlisting[language=Java, tabsize = 4, caption={\emph{Tipos de tokens definidos: palavra reservada, identificador, n�mero, final de arquivo e cadeia de palavras.}}, label = {code:codigos_TokenType}]{codigos/TokenType.java}


A constru��o do l�xico utilizou os c�digos apresentados na figura \ref{fig:figuras_lexico} e sua classe principal -- que chama as fun��es dos outros c�digos -- encontra-se no c�digo \ref{code:anallex}.

\inputpng{figuras/lexico}{figuras_lexico}{C�digos utilizados para construir o analisador l�xico}{1.0}

\lstinputlisting[language=Java, tabsize = 4, caption={\emph{Classe ``Analyzer.java'' do pacote ``lex'', controla o processo de an�lise l�xica.}}, label = {code:anallex}]{codigos/Analyzerlex.java}
\chapter{Analise sint�tica}
	A an�lise Sint�tica se utiliza de um automato estruturado de pilha que � descrito por \cite{JJ}, estrutura que utiliza uma pilha e alguns automatos finitos para realizar a tarefa de an�lise sint�tica.

\subsection{Estrutura}
O analisador sint�tico possu� basicamente um automato estruturado de pilha e acesso � c�digos que permitem o parse de arquivos de configura��o dos automatos finitos.

O Automato estruturado de pilha, tem sua estrutura definida por :
\subsubsection{Automato Finito}
O automato finito utilizado para a an�lise sint�tica possu� um vetor de booleanos que denota os seus estados, se uma posi��o do vetor possuir o booleano verdadeiro, quer dizer que aquele estado � final. O automato tamb�m possu� um vetor com as transi��es do automato.

\subsubsection{Transi��o} 
A transi��o pode ser de dois tipos, uma transi��o normal , ou a chamada de outro automato. Se a transi��o for normal, ela possu� o pr�ximo estado e o token que a ativa. Se a transi��o for uma chamada, a transi��o cont�m o n�mero do automato a ser chamado e o estado de retorno, que � um estado no automato atual que ser� o estado corrente ap�s a execu��o do segundo automato.

\subsubsection{Pilha}
A pilha � a estrutura utilizada pelo Automato estruturado de pilha para armazenar a informa��o de automato e estado quando a transi��o de um automato a ser executada � a chamada de outro automato. Quando esse tipo de transi��o ocorre o automato corrente passa a ser o automato denotado pela transi��o e a pilha guarda qual o automato que realizou a chamada, e qual o estado que esse deve voltar ap�s a execu��o do segunto automato.


\subsection{Parse}
A gram�tica em nota��o de Wirth foi reduzida de forma a ficarmos com 3 automatos, um de programa, um de bloco de c�digo e o terceiro de express�es. A gram�tica foi transformada em aut�matos finitos determin�sticos m�nimos com o uso da ferramenta (COLOCAR O RADIANTFIRE). Um exemplo da sa�da para um dos automatos encontra-se na se��o de anexo, c�digo  REFERENCIA!  A primeira linha do arquivo � sempre igual, indicando que o estado inicial do automato � o estado zero, a segunda linha mostra quais s�o os estados finais e as outras linhas indicam as transi��es. Se o elemento da transi��o representar um token � uma transi��o normal, caso seja o nome de um automato, indica a chamada para o outro automato.

\subsection{An�lise Sint�tica}



 
\chapter{Defini��o do ambiente de execu��o}
	O ambiente de execu��o � composto por uma cole��o de bibliotecas escritas em baixo n�vel, assembly da MVN para prover algumas funcionalidades b�sicas. S�o elas:

\begin{itemize}
	\item fun��o l�gica OU
	\item fun��o l�gica E
	\item fun��o comparativa >
	\item fun��o comparativa <
	\item fun��o comparativa >=
	\item fun��o comparativa <=
	\item fun��o comparativa ==
	\item fun��o comparativa !=
	\item fun��o l�gica !
	\item fun��o print de strings
\end{itemize}

A ideia � manter esses arquivos com essas fun��es em arquivos separados, pelo uso da biblioteca do montador, ligador e relocador, um �nico arquivo do programa principal pode ser gerado.

\subsection{Registros de Ativa��o}
A implementa��o da MVN n�o tem uma pilha em seu ambiente de execu��o, isso implica na impossibilidade de realizar chamadas de fun��o recursivamente. Um modo de fazer isso � utilizando-se de registros de ativa��o para contornar o problema.

\chapter{Estrutura de dados e algoritmos do compilador}
	% O objetivo deste projeto � modelar e construir um compilador que utilize uma linguagem de programa��o imperativa.
\chapter{Sem�ntica din�mica}
	Para a constru��o do analisador sem�ntico, foram usados os c�digos apresentados na figura \ref{fig:analissem}.

\inputpng{figuras/semantica}{analissem}{C�digos utilizados para construir o analisador sem�ntico}{1.0}

Cada uma dessas classes apresentam-se nos c�digos \ref{code:codigos_Semantic}, \ref{code:codigos_TokensList} e \ref{code:codigos_TokenStack}.

\lstinputlisting[language=Java, tabsize = 4, caption={\emph{Classe ``Semantic.java'' do pacote ``semantic'', controla o processo de an�lise sem�ntica.}}, label = {code:codigos_Semantic}]{codigos/Semantic.java}


\lstinputlisting[language=Java, tabsize = 4, caption={\emph{Classe ``TokenList.java'' do pacote ``semantic''.}}, label = {code:codigos_TokensList}]{codigos/TokensList.java}


\lstinputlisting[language=Java, tabsize = 4, caption={\emph{Classe ``TokenStack.java'' do pacote ``semantic''.}}, label = {code:codigos_TokenStack}]{codigos/TokenStack.java}

\chapter{Integra��o das rotinas sem�nticas}
 	A implementa��o do compilador permite ao usu�rio as funcionalidades de mostrar na tela um inteiro de que esteja contido em uma vari�vel, express�es aritim�ticas, considerando prioridade e express�es booleanas sem considera��o a respeito das express�es boolenas.

Infelizmente as bibliotecas do montador, relocador e lingador apresentam algum erro em sua implementa��o e acabam inserindo no c�digo bin�rio final no meio do arquivo a seguinte linha: 0000 0002 , um desvio incondicional para o endere�o 2 de mem�ria, arruinando os resultados.
%
% Refer�ncias
%
\nocite{*}
\bibliographystyle{abnt-alf} % estilo da ABNT
\bibliography{bibliografia} % arquivo bibtex
\end{document}